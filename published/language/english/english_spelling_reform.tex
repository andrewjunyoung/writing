\documentclass{article}

%% Begin package imports %%%%%%%%%%%%%%%%%%%%%%%%%%%%%%%%%%%%%%%%%%%%%%%%%%%%%%%

% Language and font encodings
\usepackage[english]{babel}
% Package for angle quotes etc.
\usepackage[T1]{fontenc}
% Set page size and margins
\usepackage[a4paper,top=3cm,bottom=2cm,left=3cm,right=3cm,marginparwidth=1.75cm]{geometry}
\usepackage[utf8]{inputenc}
\usepackage{hyperref}
\usepackage{mathptmx} % (Times new roman)-esque font

%%%%%%%%%%%%%%%%%%%%%%%%%%%%%%%%%%%%%%%%%%%%%%%%%%%%%%%%% End package imports %%
%% Begin commands %%%%%%%%%%%%%%%%%%%%%%%%%%%%%%%%%%%%%%%%%%%%%%%%%%%%%%%%%%%%%%

\newcommand{\s}[1]{$ _{#1} $}

%%%%%%%%%%%%%%%%%%%%%%%%%%%%%%%%%%%%%%%%%%%%%%%%%%%%%%%%%%%%%%%% End commands %%

\title{Phonetic writing for the english language}
\author{Andrew J. Young}

\begin{document}
\maketitle

\begin{abstract}
\end{abstract}

\section{Introduction}

Every word an author writes takes the character of the way in which it is
written. This is a principle of art, and the calligrapher, the writer, and the
poet all know it well. So does the common person. Writing a language in one
alphabet over another can be a symbol of change, solidarity, or rebellion.
Writing is at the core of all we do as people, and as much as writing is a part
of our culture, our culture is in writing.

English has been the product of a thousand years of cultural war and evolution.
We write english with the language of Britain's first empire: the roman
empire, who along with their military and culture brought their letters,
language, and writing; The spelling of english is mostly from the french
tradition, as french speakers of Normandy were the rulers of the country since
after 1066; English has since been evolved and influenced by european languages,
especially latin and greek, all through its history. It has taken its letter
sounds, spellings, words, and literary tradition from all these foreign sources
too.

From this chaotic evolution, english has become a melting pot of cultures,
spellings, and lexicon, and as a result, its spelling and pronunciation are
famously disconnected in ways most written languages are not.

There are several famous examples, from poems designed to trip up your
pronunciation, obscure surnames with rare pronunciations and origins, and the
joy of learning to pronounce english place names like Worchestershire and
Marylebone. On any page from any author in english, hundreds of ambiguities and
confusions are written down which are disambiguated only by learning from a
young age how to pronounce words that just don't fit into the rules and patterns
of the rest of the language.

In continental Europe, most countries introduced councils to regulate spelling
and pronunciation of the national language to keep writing in line with the
sound changes that happen to any language over time. Most recently in english is
the change of pronunciation of words like «Asia» and «new»: «Asia» used to be
pronounced as it is spelled, with a clearly pronounced S and I sound; «new» used
to be pronounced as an N sound followed by «you», as it still is by many british
english speakers. But, even british speakers often pronounce words like
Newfoundland and New York as exceptions in their own accents, and use north
american pronunciations instead.

English spelling is in fact so bad that english cannot even be used to write how
to \textit{actually} pronounce words without a specialized way of doing so. The
common way is to try to use english phonetics, telling somebody that preface is
pronounced PREH-fus, but expecting that a reader can pronounce "preh`` and
"fus`` the way they should be expected to. [1]

Linguists and dictionaries nearly all use the international phonetic alphabet
(IPA) to show pronunciation now, but non-linguists struggle to read IPA symbols
without extra training, and hence many dictionaries are yet to adopt the IPA.

In some languages, the very idea of asking "how do you spell that" feels like a
question with an obvious answer: you spell it how it sounds. There are also
writing systems where there may be spelling ambiguities that come from
historical spellings, but these may be rare or generally not a problem, as in
spanish and korean.

English, however, is somewhere much farther down this spectrum of ease, perhaps
around as hard to spell as french, but harder to pronounce with ease. English is
by no means the worst offending language in breaking the alphabetic principle.
Languages like tibetan and chinese have such long literary traditions that
spellings reflect long dead pronunciations, and are even harder to learn as a
modern speaker.

Looking at tibetan, or even french, is like looking at a warning sign for what
happens when a language goes too long without updating its writing system. Most
european languages updates their spellings after or since the second world war.
English is the notable outlier.

We can fix this, and stop the need for people to understand phonetic symbols
just to learn how to read a word in a dictionary. It seems like a silly place
for a meant-to-be-phonetic alphabetical writing system to be in, and yet goes
unfixed.

At their most simple, the suggested phonetic spellings in this document make way
for precise and helpful phonetic guides for average readers. Beyond this, they
present a new mind-frame that english \textit{could} be written like this, if
we come to change, and we want our written language to match our modern culture.
This is not the same as discarding its past. Our past is inherent in the
language, in the words we use, and the way we speak them. It's also inherent in
our core alphabet. Changing the way we use our alphabet is looking inside our
own culture's soul and holding it to its best standards; not changing our
identity from what we are.

\section{Lexical sets}

English has a huge number of different accents, with a vast array of
pronunciations. Miraculously, in spite of this, english speakers understand each
other across different accents with little difficulty. This is in great part
because of \textit{lexical sets} of vowels.

The way vowels are pronounced differs from region to region, but ultimately, all
these differences follow rules, and do so remarkably consistently across the
world.

English has around 27 lexical sets, and how you pronounce the vowel which each
set describes is generally a good way of determining where you come from. This
is how The New York Times tried to do so in its quiz.

For example, a new yorker will say palm and thought with the same vowel; a
londoner will say bath and palm with the same vowel. Even minor accents follow
similar rules, and knowing how to substitute vowels is how actors and linguists
learn to master other accents with ease. Knowing that in texan accents, speakers
say /a:/ instead of /aɪ/ teaches you how to pronounce any words that standard
american says with /aɪ/, like \textit{fire}, \textit{liar}, and \textit{dye}.

\section{The acute system}

To describe English pronunciation, we use \textit{the acute system}. Germanic
languages have quite similar sound rules because they all came from a common
ancestral language. So, words English vowels may seem highly different from
German, but not as much when you know the rules. One of these is called the
\textit{germanic umlaut}.

For example, in English:

mouse becomes mice
foot becomes feet
man becomes men

In German:

Fuß becomes Füße
Mus becomes Mäuse
Man becomes Männer

The first notable rule of english vowels is «short» and «long» vowels, as they
are sometimes called. Compare:

hat vs heart (a)
tin vs teen (i)
not vs naught (o)
put, good vs food (u)

English also has a vowel, schwa, which comes about when other vowels get
contracted in speech, but also at the beginning of words like «about».


English also has rules on how to pronounce vowels when they're followed by the
letter «e». Compare:

hat, hate
pet, pete (also pee)
lit, lite (also lie)
tot, tote (also toe)
run, rune

Some people may call these "short" and "long" vowels. In the acute system, we
mark the difference between these short and long vowels with an acute accent.
So:

hat is hat
hate is hät
When said with UK received pronunciation, heart is hát.

\section{Rules}

\begin{tabular}
  New spelling & Lexical set & RP & GA & Example words \\
  ù & strUt & ʌ & ʌ & cup, suck, budge, pulse, trunk, blood \\
  a, á & bAth & ɑː & æ & staff, brass, ask, dance, sample, calf \\
  á & pAlm & ɑː & ɑ & psalm, father, bra, spa, lager \\
  a & trAp & æ & æ & tap, back, badge, scalp, hand, cancel \\
  aw & mOUth & aʊ & aʊ & out, house, loud, count, crowd, cow \\
  ay & prIce & aɪ & aɪ & ripe, write, arrive, high, try, buy \\
  ár & stARt & ɑː & ɑr & far, sharp, bark, carve, farm, heart \\
  e & commA & ə & ə & catalpa, quota, vodka \\
  er & nURse & ɜː & ɜr & hurt, lurk, urge, burst, jerk, term \\
  er & lettER & ə & ər & paper, metre, calendar, stupor, succo(u)r, martyr, figure \\
  é & drEss & e & ɛ & step, neck, edge, shelf, friend, ready \\
  ér & squARE & ɛə & ɛr & care, fair, pear, where, scarce, vary \\
  éy & fAce & eɪ & eɪ & tape, cake, raid, veil, steak, day \\
  i & kIt & ɪ & ɪ & ship, sick, bridge, milk, myth, busy \\
  ir & NEAR & ɪə & ɪr & beer, sincere, fear, beard, serum \\
  í & happY & i & i & copy, scampi, taxi, sortie, committee, hockey, Chelsea \\
  í & flEEce & iː & i & creep, speak, leave, feel, key, people \\
  o, ó & clOth & ɒ & ɔ & cough, broth, cross, long, Boston \\
  o & lOt & ɒ & ɑ & stop, sock, dodge, romp, possible, quality \\
  ow & gOAt & əʊ & oʊ & soap, joke, home, know, so, roll \\
  ó & thOUght & ɔː & ɔ & taught, sauce, hawk, jaw \\
  ór & fORce & ɔː & or & four, wore, sport, porch, borne, story \\
  ór & nORth & ɔː & ɔr & for, war, short, scorch, born, warm \\
  óy & chOIce & ɔɪ & ɔɪ & adroit, noise, join, toy, royal \\
  u & fOOt & ʊ & ʊ & put, bush, full, good, look, wolf \\
  ú & gOOse & uː & u & loop, shoot, tomb, mute, huge, view \\
  ur & cURe & ʊə & ʊr & poor, tourist, pure, plural, jury \\
\end{tabular}

In general:

\begin{tabular}
  Letter & Vowel usually represented \\
  a & æ \\
  á & ɑ \\
  e & ə, ɜ \\
  é & ɛ, e \\
  i & ɪ \\
  í & i \\
  o & o, ɒ \\
  ó & ɔ \\
  u & ʊ \\
  ú & u \\
  ù & ʌ \\
  y & -ɪ \\
  w & -ʊ \\
\end{tabular}

\begin{tabular}
  Letter & Digraph & Consonant \\
  m & m & m
  n & n & n
  ń & nh & ŋ (ng)
  p & p & p
  b & b & b
  t & t & t
  d & d & d
  c & ch & tʃ (ch)
  j & j & dʒ (j)
  k & k & k
  g & g & k
  f & f & f
  v & v & v
  t́ & th & θ (th)
  d́ & dh & ð (th)
  s & s & s
  z & z & z
  ś & sh & ʃ (sh)
  ź & zh & ʒ (sh)
  x & x & x
  h & h & h
  l & l & l
  r & r & r
  y & y & j
  w & w & w
\end{tabular}

If you can't use diacritics, use «1» after the affected letter for acute, and
«u2» for ù.

\section{Examples}

« Colorless green ideas sleep furiously. »
GA: « Kùlerles grín áydíez slíp fyuríuslí. »
RP: « Kùlerles grín áydíez slíp fyuríuslí. »
GA: « Ku\s{2}lerles gri\s{1}n a\s{1}ydi\s{1}ez sli\s{1}p fyuri\s{1}usli\s{1}. »

« Shaw, those twelve beige hooks are joined if I patch a young, gooey mouth. »

Recommended:
GA: Śá, d́owz twelv béyź huks ár jóynd if áy pać e yùń, gúí mawt́.
RP: Śó, d́owz twelv béyź huks á jóynd if áy pać e yùń, gúí mawt́.

With diacritics:
GA: Shá, dhowz twelv béyź huks ár jóynd if áy pach e yùnh, gúí mawth.
RP: Shó, dhowz twelv béyź huks á jóynd if áy pach e yùnh, gúí mawth.

\end{document}

%%%%%%%%%%%%%%%%%%%%%%%%%%%%%%%%%%%%%%%%%%%%%%%%%%%%%%% End document contents %%

